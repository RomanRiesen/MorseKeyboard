\documentclass[a5paper,12pt]{article}
\usepackage{amssymb, amsfonts, fancyhdr, tikz, helvet, svg, titlesec, pdfpages, pdflscape, rotating}
\usepackage[left=2.3cm,right=2.3cm,top=3cm,headsep=2cm]{geometry}

\begin{document}
\renewcommand{\familydefault}{helvet}
\renewcommand{\familydefault}{\ttdefault}

\titleformat*{\section}{\Large \ttfamily}

\pagestyle{fancy}
\fancyhf{}

\fancyhead[C]{\includegraphics[width=1cm]{Logatronik.pdf}}
\fancyfoot[C]{\thepage}
%\fancyfoot[L]{\Large MCHID Type A [En/De]}

\renewcommand{\headrulewidth}{0pt}

\includepdf[pages=-]{FrontPage.pdf}
\newpage
\setcounter{page}{1}

    \tableofcontents{}
    \section{Welcome!}
    The Morse Code Human Interface Device let's you use the full power of \emph{morse code} on your computer;
    Be no longer slave to the over engineered keyboard and enjoy true minmalism with MCHID!\\
    Thank you for choosing Logatronik to dive into this brave new world of oneness with the digital.\\

    \section{Installation}\label{installation}
    To install the device simply connect it with a pc. The drivers should automatically be installed.
    Due to technical limitations \"A, \"O, \"U are only supported when the computer is set to the US international keyboard and might lead to unexpected results otherwise.
    \section{The Morse Code used}
    The code used is mostly the same as the ITU\footnote{International Telecommunication Union} standard, with the addition of German letters, though their support is limited (see section~\ref{installation}). Further there are no letters longer than 5 morse signs implemented (except deletion (which is anything longer than that)). The usual ``CH'' is replaced by a space, `` ''.\\
    Finally a few non-standard codes have been added (as they are longer than 5 signs) such as ``.'', ``,'', -'', ``*'' and insert, ``ENT''.
    The exact translations can be extracted from figure~\ref{fig:morseTree}.

    \begin{description}

        \item[Dot-time:] The maximal duration for a dot is about 170ms
        \item[Dash-time:] The maximal duration for a dash is three times that of a dot
        \item[Deletion:] If the length of an input sequence for a single character exceeds 6, the MCHID will delete the previous letter for any further press of the button, until there is no further input for the duration of a dash.
        \item[Space:] There are three ways to create spaces (see section~\ref{modifications} on how to select them).
            \begin{description}
            \item[Long Press:] By pressing the button longer than a dash-time a space is written.
            \item[Timed:] After the duration of 7 dots of the switch not being pressed a space is written (One can hold the button down to avoid printing anything, if this option is selected and the first one is not).
            \item[4 dashes:] Probably annoying after a while, but that's why there are other options.
            \end{description}
    \end{description}

    \begin{center}
    \label{morseTree}
    \begin{sidewaysfigure}[H]
    
\begin{tikzpicture}[level 1/.style={sibling distance=8cm}, level 2/.style={sibling distance=4cm}, level 3/.style={sibling distance=2cm}, level 4/.style={sibling distance=1cm}, level 5/.style={sibling distance=5mm}]
\node (start) {start}
child {node (E) {E} edge from parent[dashed]
    child {node (I) {I} edge from parent[dashed]
            child {node (S) {S} edge from parent[dashed]
                child {node (H) {H} edge from parent[dashed]
                    child {node (5) {5} edge from parent[dashed]}
                    child {node (4) {4} edge from parent[solid]}
                }
               child {node (V) {V} edge from parent[solid]
                    child {node (empty) {} edge from parent[dashed]}
                    child {node (3) {3} edge from parent[solid]}
                    }
            }
            child {node (U) {U} edge from parent[solid]
                child {node (F) {F} edge from parent[dashed]
                    child {node (dot) {.} edge from parent[dashed]}
                    child {node (comma) {,} edge from parent[solid]}
                }
                child {node ("U) {"U} edge from parent[solid]
                    child {node (empty) {} edge from parent[dashed]}
                    child {node (2) {2} edge from parent[solid]}
               }
            }
       }
    child {node (A) {A} edge from parent[solid]
            child {node (R) {R} edge from parent[dashed]
                child {node (L) {L} edge from parent[dashed]
                    child {node (empty) {} edge from parent[dashed]}
                    child {node (empty) {} edge from parent[solid]}
               }
                child {node ("A) {"A} edge from parent[solid]
                    child {node (sign+) {+} edge from parent[dashed]}
                    child {node (sign-) {-} edge from parent[solid]}
                }
            }
            child {node (W) {W} edge from parent[solid]
               child {node (P) {P} edge from parent[dashed]
                    child {node (empty) {} edge from parent[dashed]}
                    child {node (empty) {} edge from parent[solid]}
               }
               child {node (J) {J} edge from parent[solid]
                    child {node (empty) {} edge from parent[dashed]}
                    child {node (1) {1} edge from parent[solid]}
               }
            }
        }
    }
child {node (T) {T} edge from parent[solid]
    child {node (N) {N} edge from parent[dashed]
        child {node (D) {D} edge from parent[dashed]
            child {node (B) {B} edge from parent[dashed]
                child {node (6) {6} edge from parent[dashed]}
                child {node (=) {=} edge from parent[solid]}
            }
            child {node (X) {X} edge from parent[solid]
                child {node (/) {/} edge from parent[dashed]}
                child {node (*) {*} edge from parent[solid]}
            }
          }
          child {node (K) {K} edge from parent[solid]
            child {node (C) {C} edge from parent[dashed]
                child {node (empty) {} edge from parent[dashed]}
                child {node (enter) {ENT} edge from parent[solid]}
            }
            child {node (Y) {Y} edge from parent[solid]
                child {node (empty) {} edge from parent[dashed]}
                child {node (empty) {} edge from parent[solid]}
            }
          }
     }
    child {node (M) {M} edge from parent[solid]
        child {node (G) {G} edge from parent[dashed]
            child {node (Z) {Z} edge from parent[dashed]
                child {node (7) {7} edge from parent[dashed]}
                child {node (empty) {} edge from parent[solid]}
            }
            child {node (Q) {Q} edge from parent[solid]
                child {node (empty) {} edge from parent[dashed]}
                child {node (empty) {} edge from parent[solid]}
            }
            }
        child {node (O) {O} edge from parent[solid]
            child {node ("O) {"O} edge from parent[dashed]
                child {node (8) {8} edge from parent[dashed]}
                child {node (empty) {} edge from parent[solid]}
            }
            child {node (" ") {" "} edge from parent[solid]
                child {node (9) {9} edge from parent[dashed]}
                child {node (0) {0} edge from parent[solid]}
            }
            }
        }
    }
;

\path (start) -- (E)[dotted];
\path (start) -- (T)[dotted];

\end{tikzpicture}

        \caption{The Tree of the implemented code}
        \label{fig:morseTree}
    \end{sidewaysfigure}
    \end{center}
    \section{Modifications}\label{modifications}
    There is already support in the MCHID for some customization, though the intrigued might want to go further; your imagination is the only limit!

    \begin{description}
        \item[Change Space printing mode:] When Pin 2 is connected to ground (GND) the ``Long Press'' mode is used.\\
        When pin 3 pin is connected to ground the ``Timed" mode is activated.\\
        Both modes can be activated simultaniously, though you will need a second connector.
    \end{description}

\end{document}
